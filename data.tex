(much of the text below is taken from {\tt mnfb\_database\_users\_guide\_2009.doc} written by Nick Danz.)

\subsection{Sampling Design}

The monitoring program was designed to provide an accurate estimate of population change for forest bird species in each study area in northern Minnesota and Wisconsin (Figure 1).  The spatial extent of the study areas is large, on the order of hundreds of thousands of hectares, and each area includes a mosaic of forest stand types.  We distributed sampling locations across the forest mosaic in a stratified random manner.  A list of forest stands was created for each study area, and stands with the same stand type according to dominant tree species and stocking density were grouped into strata.  For the national forests, stands were � 16 ha (40 acres) and were identified from the individual national forest stand inventories ca. 1990.  Stands were large enough to accommodate three sampling points a minimum of 220 meters apart.  For each national forest, a number of stands were selected from each stratum so that the final proportion of stands of each stand type was equal to the proportion of forested land area of each stand type (Hanowski and Niemi 1995).  The sample of stands is therefore representative of the forest cover in each national forest.  A total of 133, 135, and 169 stands were established in 1991-1992 the Chequamegon, Chippewa, and Superior National Forests, respectively, with approximately 20 new stands added in the Superior NF in 2008.  

The sampling unit in the St. Croix and Southeast Minnesota study areas is different than in the three national forests.  Because stands in these study area are generally small (�16 ha), only one survey point could be placed in each stand.  For these study areas, a stand had to be at least 4 ha (10 acres) in size.  Stands were stratified in forest cover types in a proportional manner similar to those in the national forests with restrictions also based on access and travel time.  A total of 171 points were established in the St. Croix area and 211 in Southeast Minnesota.  All points in the St. Croix region were located on state-owned lands.  In Southeast Minnesota, 85\% of points were on state-owned land, 6\% are on county-owned land, and 9\% are on private lands.

Changes to forest cover through natural and anthropogenic disturbance have occurred on sampling locations since the beginning of the study and may have caused concomitant changes in bird populations.  Because sampling locations are permanently marked, we are able to incorporate such changes into our descriptions of bird population patterns through time. 

\subsection{Survey Design}

Point count sampling used in our program follow national and regional standards (Ralph et al. 1993, 1995, Howe et al. 1997).  Ten-minute point counts are conducted at each point between June and early July (Reynolds et al. 1980).  Point counts are appropriate for determining the relative abundance of most singing passerine species, but are inadequate for waterfowl, grouse, woodpeckers, and most raptors.  In addition, because our surveys are conducted during the summer months, we may underestimate the relative abundance of early-nesting species (e.g. permanent residents that begin breeding in April, such as woodpeckers and chickadees).

Point counts are conducted by trained observers (see observer training section below) from approximately 0.5 hour before to 4 hours after sunrise on days with little wind (< 15 km/hr) and little or no precipitation.  All birds heard or seen from the point were recorded with estimates of their distance from that point.  From 1991 to 1994, all birds heard or seen within 100 m of the point were recorded.  From 1995-2006, we included all birds heard or seen from the point regardless of distance so that our results could be compared with other monitoring programs in this region (see Howe et al. 1997).  The number of individuals observed for each species can be summed for 3, 5, and 10-minute periods so that regional comparisons are possible with data gathered using 3 or 5-minute point counts.  In 2008, the time intervals were expanded to include birds observed in an opening 2-min interval followed by eight 1-minute intervals (i.e. 0-2, 3, 4�9) to match national protocols (Knutson et al. 2008) and to allow estimation of detection heterogeneity with distance-removal methods (Etterson et al. 2009)

Each year, we attempt to have each observer sample a similar number of stands of each forest cover type.  This is done to minimize bias due to observer differences in sampling different forest cover types.  Weather data (cloud cover, temperature, and wind speed) and time of day were recorded before each count.

\subsection{Vegetation}

Vegetation characteristics at survey locations have been recorded on a number of occasions.  In the �quick vegetation protocol�, a number of measures of tree and shrub cover, density, and foliage height profile are made with ocular estimates from the point count center.  The quick protocol has been used on all site locations at least twice over the course of the study and in some cases points have been surveyed for vegetation multiple times.  In summer 2006, more in-depth vegetation surveys were carried out by crews of 2 people using measured quantites rather than ocular estimates.  The purpose of the in-depth protocol was to provide a basis of comparison to the quick protocol.  However, as of this writing, comparison of the two protocols or analysis using the in-depth protocol has yet to be completed.
