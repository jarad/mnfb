(much of the text below is taken from {\tt mnfb\_database\_users\_guide\_2009.doc} written by Nick Danz.)

Forests of the western Great Lakes region have among the richest diversity of breeding bird species in North America (Green 1995, Rich et al. 2004).  An increased appreciation of this diversity, along with concerns about potential declines of some species led to a strong interest in monitoring forest bird populations in the region.  For example, agencies such as the USDA Forest Service have a need for population trend data at the scale of an individual national forest to identify when and where population changes are occurring and to identify potential conservation problems.  In response to the need for regional population data, the Minnesota Forest Breeding Bird Project (MNFB) was established in 1991.  The overall goal of the project is to sustain forest resources and bird diversity in western Great Lakes forests.  This goal is achieved through four primary activities:

\begin{enumerate}
\item	Monitoring. An extensive, long-term monitoring program with over 1600 off-road sampling points designed to track regional population trends and investigate the response of forest birds to regional land use patterns.
\item	Research. Intensive field studies designed to describe bird-habitat relationships and identify factors responsible for observed population trends.
\item	Modeling. Use of geographic information system (GIS) techniques to spatially and temporally relate distribution and abundance of forest birds to forest habitat features at the stand and landscape levels.
\item	Education and Management. Dissemination of research findings and development of educational and management tools to promote forest bird conservation.
\end{enumerate}

Since its inception, the project has collected avian field data in five regional study areas, with each area having different start dates and sample sizes.  Surveys began in 1991 on the Chippewa and Superior National Forests in Minnesota.  In 1992, the effort was expanded to include sites on the Chequamegon National Forest in Wisconsin and the St. Croix River region of east-central Minnesota.  In 1992, sites were also added in the Superior NF.  In 1995, the project was further expanded to include southeastern Minnesota, although surveys on these sites were discontinued in 2001 due to funding cuts.  Surveys in the St. Croix region were similarly discontinued in 2003 due to funding cuts.  In 2008, an additional 75 sites were added in primarily lowland conifer stands in the Superior National Forest to add to the gain greater representation of that forest type.  As of 2013, sites were still being surveyed in the three national forests.

Project staff provides an annual updates report as well as many other summaries and analyses at the project website: http://www.nrri.umn.edu/mnbirds.
