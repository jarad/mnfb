\subsection{Sites}

Over the lifetime of the study (1991-2012), observers have surveyed 1429 sites grouped into 461 stands in the three study forests (Table \ref{tab:stands-sites-per-forest}).  As Figure \ref{fig:site-siteyear} demonstrates, most sites have been surveyed every year during that time, but many have been omitted for the occasional year or two.  Quite a few were discontinued around 1994-1995, and a significant number were introduced in Superior Forest in 2008.  It can also be seen from the veritical white bands in Figure \ref{fig:site-siteyear} that some years have unusually high missing counts, which presumably result from either weather or administrative causes.\par
There are few indicators of anything amiss within the sites data.  75 sites do not have Lat-Lon coordinates (Table \ref{tab:site-duplicated}).  There is one set of bird counts in the Birds data table (24 species observed) that has no corresponding observation conditions in the Site data table.  Conversely, there are three site-years in the Site data table for which no birds of any species were counted in the Birds data table (Table \ref{tab:site-missing-data}).

\begin{figure}
\includegraphics{site-siteyear}
\caption{For each forest, depicts whether a site was surveyed (black) or not (white).}
\label{fig:site-siteyear}
\end{figure}

\input{tables/stands-sites-per-forest}

\input{tables/site-duplicated}

\input{tables/site-missing-data}

\begin{figure}
\includegraphics{site-map}
\caption{A map of the sites}
\label{fig:site-map}
\end{figure}

Figure \ref{fig:SiteDatesPlot-byForestbyYear} illustrates that the three forests are sampled in the same order every year.  As we will discuss later, this is one reason to model bird counts in each forest separately.  Otherwise, seasonal effects (julian dates) get confounded with location.

\begin{figure}
\includegraphics{SiteDatesPlot-byForestbyYear}
\caption{Sampling Dates at Each Forest 1995-2012}
\label{fig:SiteDatesPlot-byForestbyYear}
\end{figure}

Figure \ref{fig:SiteDatesPlot-byForestbyYear} suggests that several observations in the site database table may be miscoded for date (or perhaps for site).  For each of the following cases (Table \ref{tab:isolated-observations}), only 1 or 2 observations were recorded in a particular forest on the date listed, and no other observations were made in that forest within 2 days:

\input{tables/SiteDatesPlot-isolated-observations}

\begin{figure}
\includegraphics{SiteDatesPlot-orderwithinCheq}
\caption{Map of Sampling Dates at Each Forest 1995-2012}
\label{fig:SiteDatesPlot-orderwithinCheq}
\end{figure}

\begin{figure}
\includegraphics{SiteDatesPlot-orderwithinChip}
\caption{Map of Sampling Dates at Each Forest 1995-2012}
\label{fig:SiteDatesPlot-orderwithinChip}
\end{figure}

\begin{figure}
\includegraphics{SiteDatesPlot-orderwithinSup}
\caption{Map of Sampling Dates at Each Forest 1995-2012}
\label{fig:SiteDatesPlot-orderwithinSup}
\end{figure}

In the same way that the sampling order of forests is the same every year, the sampling order of sites \textit{within} a forest is also the same every year with a few exceptions (Figs. \ref{fig:SiteDatesPlot-orderwithinCheq}, \ref{fig:SiteDatesPlot-orderwithinChip}, and \ref{fig:SiteDatesPlot-orderwithinSup}).  Again, this causes seasonal effects (julian dates) to get confounded with location -- only, this time the location is at the site-stand level instead of at the forest level.


observations without sites
sites with observations



\subsection{Observers}

THIS SECTION ON HOLD UNTIL OBSERVER ID RESOLVED

Patterns in observer turnover influence the analysis of the dataset (Fig. \ref{fig:Observers-obsyear}).  In most years, from 5 to 8 observers collect bird counts.  However, 61\% of observers work for only one field season.  These one-season observers account for 35.8\% of all site observations; two-year observers account for an additional 23.0\% of all site observations (Table \ref{tab:field-seasons}).  [These summaries use data back to 1991; however, we are only attempting to model data from 1995 onward].

\input{tables/Observers-field-seasons}

In particular, these patterns affect two questions.  First, how well can we measure an observer effect separate from annual variation?  Because many observers only work one field season, the variation among observers is somewhat confounded with variation from year to year.  Second, how consistent is an observer's effect from year to year?  In particular, is there a first-year effect for observers?  Necessarily, the data from single-season observers cannot address these questions.

\subsubsection{Potential Observer Anomalies}

1 - Large gaps in service
2 - Low-count years
3 - Anything else fishy

\begin{figure}
\includegraphics{Observers-obsyear.pdf}
\caption{Years of Data Collection for Each Observer}
\label{fig:Observers-obsyear}
\end{figure}





\subsection{Time}

Time of Day
- insignificance of sunrise (12 minutes)





\subsection{Observation Conditions}

Sky, Wind, Noise
- quadratic annual trend in "noise"





\subsection{Habitat}

Habitat Structure
- Variable selection
  - lack of clarity
    - which to use
    - various definitions of 'regeneration'
- Structure of variables
- Change over time
- Distribution by forest (is Superior weird?)
- Stand homogeneity

- modeling impact:
  - nested random versus fixed
  - imbalanced 'design'